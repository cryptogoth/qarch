% a sample file for Journal of Quantum Information and Computation (QIC) in 
% LaTex2e by inputing macro file "qic.sty" with command \usepackage{qic}, 
% all the macros have been defined in the style file, so it is no need to 
% put many macros at the beginning of the text file  

\documentclass[twoside]{article}
%\usepackage[dvips]{graphicx}
%\usepackage{times}
%\usepackage{fullpage}
%\usepackage{rotating}
%\usepackage{eepic}
%\usepackage{amsfonts}
%\usepackage{algorithmic}
%\usepackage{amsthm}

%\theoremstyle{plain}
%\newtheorem{theorem}{Theorem}

\newcommand{\targfix}{\qw {\xy {<0em,0em> \ar @{ - } +<.39em,0em>
\ar @{ - } -<.39em,0em> \ar @{ - } +<0em,.39em> \ar @{ - }
-<0em,.39em>},<0em,0em>*{\rule{.01em}{.01em}}*+<.8em>\frm{o}
\endxy}}

\input{Qcircuit}

\newcommand{\braket}[2]{\langle #1|#2 \rangle}
\newcommand{\normtwo}{\frac{1}{\sqrt{2}}}
\newcommand{\norm}[1]{\parallel #1 \parallel}

\newcommand\coolleftbrace[2]{%
#1\left\{\vphantom{\begin{matrix} #2 \end{matrix}}\right.}

\begin{document}

%%%%%%%%%%%%%%%%%%%%%%%%%%%%%%%%%%%%%%%%%%%%%%%%%%%%%%%%%%%%%%%%%%%%%%%%%%%%%%%
% 1
\begin{figure}
\begin{displaymath}
\Qcircuit @C=2em @R=2em {
\lstick{\ket{0}} & \qw        & \qw & \qw \\
\lstick{\ket{a_i}} & \qw      & \qw & \qw \\
\lstick{\ket{b_i}} & \ctrl{1} & \qw & \qw \\
\lstick{\ket{c_i}} & \ctrl{1} & \qw & \qw \\
\lstick{\ket{0}} & \targfix   & \qw & \qw_{\ket{b_i \wedge c_i}} 
}
\end{displaymath}
\caption{Step 1}
\end{figure}

%%%%%%%%%%%%%%%%%%%%%%%%%%%%%%%%%%%%%%%%%%%%%%%%%%%%%%%%%%%%%%%%%%%%%%%%%%%%%%%
% 2
\begin{figure}
\begin{displaymath}
\Qcircuit @C=2em @R=2em {
\lstick{\ket{0}} & \qw      & \qw & \qw                 & \qw & \qw                          \\
\lstick{\ket{a_i}} & \qw      & \qw & \qw                 & \qw & \qw                        \\
\lstick{\ket{b_i}} & \ctrl{1} & \qw & \targfix            & \qw & \qw_{\ket{b_i \oplus c_i}} \\
\lstick{\ket{c_i}} & \ctrl{1} & \qw & \ctrl{-1}           & \qw & \qw                        \\
\lstick{\ket{0}} & \targfix & \qw & \qw_{\ket{b_i \wedge c_i}} & \qw & \qw                   
}
\end{displaymath}
\caption{Step 2}
\end{figure}

%%%%%%%%%%%%%%%%%%%%%%%%%%%%%%%%%%%%%%%%%%%%%%%%%%%%%%%%%%%%%%%%%%%%%%%%%%%%%%%
% 3
\begin{figure}
\begin{displaymath}
\Qcircuit @C=2em @R=2em {
\lstick{\ket{0}} & \qw      & \qw & \qw                 & \qw & \qw                        & \targfix  & \qw & \qw_{\ket{a_i \wedge (b_i \oplus c_i)}} \\
\lstick{\ket{a_i}} & \qw      & \qw & \qw                 & \qw & \qw                        & \ctrl{-1} & \qw & \qw                              \\
\lstick{\ket{b_i}} & \ctrl{1} & \qw & \targfix            & \qw & \qw_{\ket{b_i \oplus c_i}} & \ctrl{-1} & \qw & \qw                              \\
\lstick{\ket{c_i}} & \ctrl{1} & \qw & \ctrl{-1}           & \qw & \qw                        & \qw       & \qw & \qw                              \\
\lstick{\ket{0}} & \targfix & \qw & \qw_{\ket{b_i \wedge c_i}} & \qw & \qw                        & \qw       & \qw & \qw                         
}
\end{displaymath}
\caption{Step 3}
\end{figure}

%%%%%%%%%%%%%%%%%%%%%%%%%%%%%%%%%%%%%%%%%%%%%%%%%%%%%%%%%%%%%%%%%%%%%%%%%%%%%%%
% 5
\begin{figure}
\begin{displaymath}
\Qcircuit @C=2em @R=2em {
\lstick{\ket{0}} & \qw      & \qw & \qw                 & \qw & \qw                        & \targfix  & \qw & \qw_{\ket{a_i \wedge (b_i \oplus c_i)}} & \targfix  & \qw       & \qw & \qw_{\ket{(b_i \wedge c_i) \oplus a_i \wedge (b_i \oplus c_i)}} \\
\lstick{\ket{a_i}} & \qw      & \qw & \qw                 & \qw & \qw                        & \ctrl{-1} & \qw & \qw                                   & \qw       & \targfix  & \qw & \qw_{\ket{a_i \oplus b_i \oplus c_i}} \\
\lstick{\ket{b_i}} & \ctrl{1} & \qw & \targfix            & \qw & \qw_{\ket{b_i \oplus c_i}} & \ctrl{-1} & \qw & \qw                                   & \qw       & \ctrl{-1} & \qw & \qw \\
\lstick{\ket{c_i}} & \ctrl{1} & \qw & \ctrl{-1}           & \qw & \qw                        & \qw       & \qw & \qw                                   & \qw       & \qw       & \qw & \qw \\
\lstick{\ket{0}} & \targfix & \qw & \qw_{\ket{b_i \wedge c_i}} & \qw & \qw                        & \qw       & \qw & \qw                              & \ctrl{-4} & \qw       & \qw & \qw
}
\end{displaymath}
\caption{Step 5}
\end{figure}

%%%%%%%%%%%%%%%%%%%%%%%%%%%%%%%%%%%%%%%%%%%%%%%%%%%%%%%%%%%%%%%%%%%%%%%%%%%%%%%
% 6
\begin{figure}
\begin{displaymath}
\Qcircuit @C=2em @R=2em {
\lstick{\ket{0}} & \qw      & \qw & \qw                 & \qw & \qw                        & \targfix  & \qw & \qw_{\ket{a_i \wedge (b_i \oplus c_i)}} & \targfix  & \qw       & \qw       & \qw_{\ket{(b_i \wedge c_i) \oplus a_i \wedge (b_i \oplus c_i)}} \\
\lstick{\ket{a_i}} & \qw      & \qw & \qw                 & \qw & \qw                        & \ctrl{-1} & \qw & \qw                                   & \qw       & \targfix  & \qw       & \qw_{\ket{a_i \oplus b_i \oplus c_i}} \\
\lstick{\ket{b_i}} & \ctrl{1} & \qw & \targfix            & \qw & \qw_{\ket{b_i \oplus c_i}} & \ctrl{-1} & \qw & \qw                                   & \qw       & \ctrl{-1} & \targfix  & \qw \\
\lstick{\ket{c_i}} & \ctrl{1} & \qw & \ctrl{-1}           & \qw & \qw                        & \qw       & \qw & \qw                                   & \qw       & \qw       & \ctrl{-1} & \qw \\
\lstick{\ket{0}} & \targfix & \qw & \qw_{\ket{b_i \wedge c_i}} & \qw & \qw                        & \qw       & \qw & \qw                              & \ctrl{-4} & \qw       & \qw       & \qw
}
\end{displaymath}
\caption{Step 6}
\end{figure}

%%%%%%%%%%%%%%%%%%%%%%%%%%%%%%%%%%%%%%%%%%%%%%%%%%%%%%%%%%%%%%%%%%%%%%%%%%%%%%%
% 8
\begin{figure}
\begin{displaymath}
\Qcircuit @C=2em @R=2em {
\lstick{\ket{0}} & \qw      & \qw & \qw                 & \qw & \qw                        & \targfix  & \qw & \qw_{\ket{a_i \wedge (b_i \oplus c_i)}} & \targfix  & \qw       & \qw    & \qw_{\ket{(b_i \wedge c_i) \oplus a_i \wedge (b_i \oplus c_i)}}      & \qw & \rstick{\ket{v_{i+1}}}\\
\lstick{\ket{a_i}} & \qw      & \qw & \qw                 & \qw & \qw                        & \ctrl{-1} & \qw & \qw                             & \qw       & \targfix  & \qw  & \qw_{\ket{a_i \oplus b_i \oplus c_i}}  & \qw & \rstick{\ket{u_i}} \\
\lstick{\ket{b_i}} & \ctrl{1} & \qw & \targfix            & \qw & \qw_{\ket{b_i \oplus c_i}} & \ctrl{-1} & \qw & \qw                             & \qw       & \ctrl{-1} & \targfix  & \ctrl{1} & \qw & \rstick{\ket{b_i}} \\
\lstick{\ket{c_i}} & \ctrl{1} & \qw & \ctrl{-1}           & \qw & \qw                        & \qw       & \qw & \qw                             & \qw       & \qw       & \ctrl{-1} & \ctrl{1} & \qw & \rstick{\ket{c_i}} \\
\lstick{\ket{0}} & \targfix & \qw & \qw_{\ket{b_i \wedge c_i}} & \qw & \qw                        & \qw       & \qw & \qw                             & \ctrl{-4} & \qw       & \qw       & \targfix & \qw & \rstick{\ket{0}}
}
\end{displaymath}
\caption{Step 8}
\end{figure}

\begin{displaymath}
\left\{
\begin{array}{ccccc}
    & a_3 & a_2 & a_1 & a_0 \\
    & b_3 & b_2 & b_1 & b_0 \\
    & c_3 & c_2 & c_1 & c_0 \\
\hline
    & u_3 & u_2 & u_1 & u_0 \\
v_4 & v_3 & v_2 & v_1 &    \\
\hline
x_4 & x_3 & x_2 & x_1 & x_0
\end{array}
\right\}
=
\left\{
\begin{array}{ccccc}
    & 1 & 0 & 1 & 1 \\
    & 1 & 0 & 1 & 0 \\
    & 1 & 0 & 0 & 1 \\
\hline
    & 1 & 0 & 0 & 0 \\
  1 & 0 & 1 & 1 &   \\
\hline
1 & 1 & 1 & 1 & 0
\end{array}
\right\}
=
\left\{
\begin{array}{r}
11 \\
10 \\
9 \\
\hline
8 \\
22 \\
\hline
30
\end{array}
\right\}
\end{displaymath}

\begin{displaymath}
\left\{
\begin{array}{ccccc}
    & 1 & 1 & 1 & 1 \\
    & 1 & 1 & 1 & 1 \\
    & 1 & 1 & 1 & 1 \\
\hline
    & 1 & 1 & 1 & 1 \\
  1 & 1 & 1 & 1 &   \\
\end{array}
\right\}
=
\left\{
\begin{array}{r}
15 \\
15 \\
15 \\
\hline
15 \\
30 \\
\end{array}
\right\}
\end{displaymath}

\end{document}