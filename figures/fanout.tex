% a sample file for Journal of Quantum Information and Computation (QIC) in 
% LaTex2e by inputing macro file "qic.sty" with command \usepackage{qic}, 
% all the macros have been defined in the style file, so it is no need to 
% put many macros at the beginning of the text file  

\documentclass[twoside]{article}
%\usepackage[dvips]{graphicx}
%\usepackage{times}
%\usepackage{fullpage}
%\usepackage{rotating}
%\usepackage{eepic}
%\usepackage{amsfonts}
%\usepackage{algorithmic}
%\usepackage{amsthm}

%\theoremstyle{plain}
%\newtheorem{theorem}{Theorem}

\newcommand{\targfix}{\qw {\xy {<0em,0em> \ar @{ - } +<.39em,0em>
\ar @{ - } -<.39em,0em> \ar @{ - } +<0em,.39em> \ar @{ - }
-<0em,.39em>},<0em,0em>*{\rule{.01em}{.01em}}*+<.8em>\frm{o}
\endxy}}

\input{Qcircuit}

\newcommand{\braket}[2]{\langle #1|#2 \rangle}
\newcommand{\normtwo}{\frac{1}{\sqrt{2}}}
\newcommand{\norm}[1]{\parallel #1 \parallel}

\begin{document}

\begin{displaymath}
\Qcircuit @C=.7em @R=1.2em {
\lstick{\ket{\psi}} & \qw  & \ctrl{4} & \qw & \ctrl{2} & \qw & \ctrl{1} & \qw \\
                    & \qw  & \qw      & \qw & \qw      & \qw & \targfix & \qw \\
                    & \qw  & \qw      & \qw & \targfix & \qw & \ctrl{1} & \qw \\
                    & \qw  & \qw      & \qw & \qw      & \qw & \targfix & \qw \\
                    & \qw  & \targfix & \qw & \ctrl{2} & \qw & \ctrl{1} & \qw \\
                    & \qw  & \qw      & \qw & \qw      & \qw & \targfix & \qw \\
                    & \qw  & \qw      & \qw & \targfix & \qw & \ctrl{1} & \qw \\
                    & \qw  & \qw      & \qw & \qw      & \qw & \targfix & \qw
}
\end{displaymath}

\end{document}