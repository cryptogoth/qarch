% Decomposition of controlled-R_z gate into CNOT and single-qubit rotations.

\documentclass[twoside]{article}
%\usepackage[dvips]{graphicx}
%\usepackage{times}
%\usepackage{fullpage}
%\usepackage{rotating}
%\usepackage{eepic}
%\usepackage{amsfonts}
%\usepackage{algorithmic}
%\usepackage{amsthm}

%\theoremstyle{plain}
%\newtheorem{theorem}{Theorem}

\newcommand{\targfix}{\qw {\xy {<0em,0em> \ar @{ - } +<.39em,0em>
\ar @{ - } -<.39em,0em> \ar @{ - } +<0em,.39em> \ar @{ - }
-<0em,.39em>},<0em,0em>*{\rule{.01em}{.01em}}*+<.8em>\frm{o}
\endxy}}

\input{Qcircuit}

\newcommand{\braket}[2]{\langle #1|#2 \rangle}
\newcommand{\normtwo}{\frac{1}{\sqrt{2}}}
\newcommand{\norm}[1]{\parallel #1 \parallel}

\newcommand\coolleftbrace[2]{%
#1\left\{\vphantom{\begin{matrix} #2 \end{matrix}} \right.}

\newcommand\coolrightbrace[2]{%
\left. \vphantom{\begin{matrix} #2 \end{matrix}} \right\} #1}

\begin{document}

\begin{displaymath}
\Qcircuit @C=1.5em @R=1.5em {
\lstick{\ket{\psi}}                  & \qw & \targfix  & \qw & \gate{X^j} & \qw & \rstick{R_Z((-1)^j \phi)\ket{\psi}} \\
 %\normtwo \left( \ket{0} + e^{i\phi}\ket{1} \right) & \qw & \ctrl{-1} & \qw & \measureD  & \cw & \rstick{j}
\lstick{ \normtwo \left( \ket{0} + e^{i\phi}\ket{1} \right) } & \qw & \ctrl{-1} & \qw & \measureD{Z}  & \cw & \rstick{j}
 }
\end{displaymath}

\end{document}